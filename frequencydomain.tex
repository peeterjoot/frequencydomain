%
% Copyright � 2016 Peeter Joot.  All Rights Reserved.
% Licenced as described in the file LICENSE under the root directory of this GIT repository.
%
%{
\input{../latex/blogpost.tex}
\renewcommand{\basename}{frequencydomain}
%\renewcommand{\dirname}{notes/phy1520/}
\renewcommand{\dirname}{notes/ece1228-electromagnetic-theory/}
%\newcommand{\dateintitle}{}
%\newcommand{\keywords}{}

\input{../latex/peeter_prologue_print2.tex}

\usepackage{peeters_layout_exercise}
\usepackage{peeters_braket}
\usepackage{peeters_figures}
\usepackage{siunitx}
\usepackage{macros_cal}
%\usepackage{mhchem} % \ce{}
%\usepackage{macros_bm} % \bcM
%\usepackage{macros_qed} % \qedmarker
%\usepackage{txfonts} % \ointclockwise

\beginArtNoToc

\generatetitle{Geometric Algebra representation of Maxwell's equations in the frequency domain}
%\chapter{Geometric Algebra representation of Maxwell's equations in the frequency domain}
%\label{chap:frequencydomain}
% \citep{doran2003gap}
% \citep{jackson1975cew}
% \citep{griffiths1999introduction}

\section{Geometric Algebra form of Maxwell's equations in the frequency domain}

Presuming time dependence of all fields and sources of the form

\begin{dmath}\label{eqn:frequencydomain:20}
\calT(\Bx, t) = \Real( T(\Bx) e^{j \omega t} ),
\end{dmath}

the time harmonic form of Maxwell's equations are

\begin{dmath}\label{eqn:frequencydomain:40}
\begin{aligned}
\spacegrad \cross \BE &= -j \omega \BB \\
\spacegrad \cdot \BD &= \rho \\
\spacegrad \cross \BH &= \BJ + j \omega \BD \\
\spacegrad \cdot \BB &= 0.
\end{aligned}
\end{dmath}

In linear simple media, with \( \BD = \epsilon \BE \), and \( \BB = \mu \BH \), the Geometric Algebra identity

\begin{dmath}\label{eqn:frequencydomain:120}
\Bx \By = \Bx \cdot \By + I \Bx \cross \By,
\end{dmath}

where \( I = \Be_1 \Be_2 \Be_3 \) is the \R{3} pseudoscalar, allows Maxwell's frequency domain equations to be reduced to a pair of coupled multivector equations

\begin{dmath}\label{eqn:frequencydomain:60}
\begin{aligned}
\spacegrad \BE     &= -j \omega \mu (I \BH) + \rho/\epsilon \\
\spacegrad (I \BH) &= -\BJ - j \omega \epsilon \BE.
\end{aligned}
\end{dmath}

Furthermore,
introducing an electromagnetic field

\begin{dmath}\label{eqn:frequencydomain:80}
F = \BE + \eta I \BH,
\end{dmath}

where \( \eta = \sqrt{\mu/\epsilon} \) is the impedance of the media, a further consolidation is possible

\begin{dmath}\label{eqn:frequencydomain:100}
(\spacegrad + j k) F
=
\rho/\epsilon - \eta \BJ.
\end{dmath}

Here \( k = \omega/v \) is the wave number, and \( v = 1/\sqrt{\mu\epsilon} \) is the group velocity of a wave in the media.  Observe that there are two complex quantities in the mix \( j = \sqrt{-1} \), and \( I^2 = -1 \), each of which commutes with any of the fields or sources, or the gradient operator.

\section{Plane waves}

For source free conditions, and plane waves of the form \( \BE = \BE_0 e^{-j \Bk \cdot \Bx}, \BH = \BH_0 e^{-j \Bk \cdot \Bx} \), Maxwell's \cref{eqn:frequencydomain:100} is reduced to a single homogeneous first order multivector equation

\begin{dmath}\label{eqn:frequencydomain:140}
(\spacegrad + j k) F = 0.
\end{dmath}

The gradient action on the electromagnetic field, with summation convention implied, is

\begin{dmath}\label{eqn:frequencydomain:160}
\spacegrad F_0 e^{-j \Bk \cdot \Bx}
=
\Be_m \partial_m
F_0 e^{-j \Bk \cdot \Bx}
=
\Be_m 
F_0
\lr{ -j k_m }
e^{-j \Bk \cdot \Bx}
=
-j \Bk F_0,
\end{dmath}

so

\begin{dmath}\label{eqn:frequencydomain:180}
j k (1 - \kcap) F_0 = 0.
\end{dmath}

This means that the field must be of the form

\begin{dmath}\label{eqn:frequencydomain:200}
F = (1 + \kcap) \BE_0 e^{-j \Bk \cdot \Bx},
\end{dmath}

where \( \BE_0 \) is a vector valued complex constant, and \( \kcap \cdot \BE_0 = 0 \).  The dot product constraint follows from the requirement that the \( I \BH \) portion of the electromagnetic field is a bivector.

From \cref{eqn:frequencydomain:200} the interdependencies of the electric and magnetic field portions of the field can be read off immediately.  Those are

\begin{dmath}\label{eqn:frequencydomain:220}
\begin{aligned}
\BE &= \BE_0 e^{-j \Bk \cdot \Bx} \\
I \BH &= \inv{\eta} \kcap \BE_0 e^{-j \Bk \cdot \Bx}.
\end{aligned}
\end{dmath}

Observe that it must also be the case that the magnetic field is perpendicular to the propagation direction

\begin{dmath}\label{eqn:frequencydomain:n}
\kcap \cdot \BH 
= \gpgradezero{ \kcap (-I \kcap \BE_0) } e^{-j \Bk \cdot \Bx} 
= \gpgradezero{ -I \BE_0 } e^{-j \Bk \cdot \Bx} 
= 0.
\end{dmath}

In conventional vector treatments of electromagnetic field theory the magnetic field relationship in \cref{eqn:frequencydomain:220} is written out as a vector cross product relationship instead of a bivector equation

\begin{dmath}\label{eqn:frequencydomain:240}
\BH
= -I \inv{\eta} \kcap \BE
= -I \inv{\eta} (\kcap \wedge \BE)
= -I \inv{\eta} I (\kcap \cross \BE)
= \inv{\eta} \kcap \cross \BE.
\end{dmath}

%}
%\EndArticle
\EndNoBibArticle
