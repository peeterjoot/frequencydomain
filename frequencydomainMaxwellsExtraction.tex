%
% Copyright © 2016 Peeter Joot.  All Rights Reserved.
% Licenced as described in the file LICENSE under the root directory of this GIT repository.
%

With two complex quantities in the field multivector 
%\cref{eqn:frequencydomainCore:80}, 
it might be unclear how to recover the separate electric and magnetic fields.
While \( F \) is complex with respect to both \( j \) and \( I \), this multivector has two different grades.  Specifically, \( \BE \) is a vector and \( I \BH \) is a bivector, and can both be extracted using grade selection operations.  Suppose for example

\begin{dmath}\label{eqn:frequencydomainCore:460}
\begin{aligned}
\BE &= e_1 e^{-j k z} \\
\BH &= \inv{\eta} e_2 e^{-j k z}
\end{aligned}
\end{dmath}

The composite field is

\begin{equation}\label{eqn:frequencydomainCore:480}
F = (\Be_1 + \Be_3 \Be_1) e^{-j k z}.
\end{equation}

Extracting \( \BE\), and \( \BH \) when desired requires grade 1 and grade 2 selections respectively

\begin{subequations}
\label{eqn:frequencydomainCore:500}
\begin{dmath}\label{eqn:frequencydomainCore:520}
\BE = \gpgradeone{F} = \Be_1 e^{-j k z} \\
\end{dmath}
\begin{dmath}\label{eqn:frequencydomainCore:540}
\BH
= -\frac{I}{\eta} \gpgradetwo{F}
= -\inv{\eta} (\Be_1 \Be_2 \Be_3) \Be_3 \Be_1 e^{-j k z}
= \inv{\eta} \Be_2 e^{-j k z}.
\end{dmath}
\end{subequations}

The key here is that the spatial pseudoscalar, while it satisfies \( I^2 = -1 \) like a complex number, also has a geometric structure and can be distinguished from the scalar complex \( j \) that is used in the phasor construction.
