
Let us consider the sourceless case
and multiply
\cref{eqn:frequencydomainCore:561}
$\sqrt{\epsilon}$ and the second one by  $\sqrt{\mu}$. Introducing the vectors
$\Be, \tilde{\Bh}$,
which have the same dimensions (see jancewicz),
%
\begin{dmath}\label{eqn:frequencydomainSameUnits:62}
\begin{aligned}
 \Be     &=  \sqrt{\epsilon} \, \BE\\
\tilde{\Bh} &= I \sqrt{\mu}\,  \BH.
\end{aligned}
\end{dmath}
%
we can rewrite \cref{eqn:frequencydomainCore:60} as

\begin{dmath}\label{eqn:frequencydomainSameUnits:64}
\begin{aligned}
\spacegrad \Be     &= -j k \tilde{\Bh} \\
\spacegrad \tilde{\Bh} &=  - j k \Be \, .
\end{aligned}
\end{dmath}

Furthermore, introducing the multivector electromagnetic fields $a, b$ defined as
%
\begin{dmath}\label{eqn:frequencydomainSameUnits:66}
\begin{aligned}
a &= \Be + \tilde{\Bh} \\
b &= \Be - \tilde{\Bh}
\end{aligned}
\end{dmath}
%
by summing and subtracting \cref{eqn:frequencydomainSameUnits:64} we have
%
\begin{dmath}\label{eqn:frequencydomainSameUnits:68}
\begin{aligned}
\spacegrad a &= - j k a \\
\spacegrad b &=  j k b
\end{aligned}
\end{dmath}
%
Each of the quantities $\Be, \tilde{\Bh}, a, b$ can be represented as a Pauli matrix. The electric and magnetic fields can be easily recovered from the knowledge of $a, b$.

